% Macro's for picture-based quizzes
%
% Use the following macros for a picture-quiz section
%
% \witqstart{abbrev}{name-of-section}
%   Resets the quiz internals and starts numbering from 1.
%   @param name-of-section is used in slide titles
%   @param abbrev is used as prefix for image file names
%
% \witq{answer}{ans.1}{ans.2}{ans.3}{ans.4}{ans.5}
%   Produces a question slide using *abbrev*-*answer*-q as image
%   and the five answers as text. The correct answer is *ans.1*.
%   The answers are shown in random order.
%
% \textq{question-phrase}{ans.1}..{ans.5}
%   Produces a question slide with *question-text* as the
%   actual question, and 5 answers. The correct answer is *ans.1*.
%   The answers are shown in random order.
%
% \witqend
%   Produces answers slides
%   - uses *abbrev*-*answer* as (full-screen) image for the answer 
%     if the question was made with \witq
%   - repeats the question if \txtq was used.
%
% Images may be any format supported by \includegraphics

\makeatletter

\def\wititle#1#2{
\begin{frame}
    \frametitle{}
    \begin{center}
        \Huge #1\\
        \Large #2
    \end{center}
\end{frame}
}

\newcommand*\answers{}
\newcommand*\answerimages{}
\newcounter{@qnum}

\NewList{@witqAnswers}

\def\witqstart#1#2{
    \def\@witqprefix{#1}
    \def\@witqsection{#2}
    \renewcommand\answers{}
    \renewcommand\answerimages{}
    \setcounter{@qnum}{0}
}

\def\txtq#1#2#3#4#5#6{
    \ClearList{@witqAnswers}
    \InsertRandomItem{@witqAnswers}{#2}
    \InsertRandomItem{@witqAnswers}{#3}
    \InsertRandomItem{@witqAnswers}{#4}
    \InsertRandomItem{@witqAnswers}{#5}
    \InsertRandomItem{@witqAnswers}{#6}
    \@witqFromAnswer{#1}{#2}
    \stepcounter{@qnum}
    \edef\@n{\arabic{@qnum}}%
    \edef\@arg{{#1}{\@witqCorrectResponse}{\@n}}
    \expandafter\@txtqG\@arg
}

\def\witq#1#2#3#4#5#6{
    \ClearList{@witqAnswers}
    \InsertRandomItem{@witqAnswers}{#2}
    \InsertRandomItem{@witqAnswers}{#3}
    \InsertRandomItem{@witqAnswers}{#4}
    \InsertRandomItem{@witqAnswers}{#5}
    \InsertRandomItem{@witqAnswers}{#6}
    \expandafter\@witqFromAnswer{
        \vspace{-1cm}
        \begin{center}
        \noindent\includegraphics[width=\textwidth]{img/\@witqprefix-#1-q}
        \end{center}
}{#2}

    \stepcounter{@qnum}
    \edef\@n{\arabic{@qnum}}%
    \edef\@arg{{#1}{\@witqCorrectResponse}{\@n}}
    \expandafter\@witqG\@arg
}

% Internal implementation of frame content for questions.
% - The list of answers is passed via the global list @witqAnswers
% - The question is #1
% - The correct answer (given first to \@txtq) is #2
\def\@witqFromAnswer#1#2{
    \edef\@witqFirst{#2}
    \def\@witqCorrectResponse{bogus}
    \begin{frame}
    \frametitle{\@witqsection}
    \begin{columns}
    \begin{column}{0.7\textwidth}
        #1
    \end{column}
    \begin{column}{0.3\textwidth}
        \bigskip
        \begin{enumerate}
        \setcounter{enumi}{1}
        \ForEachFirstItem{@witqAnswers}{@witqItem}{\item[\Alph{enumi}] \@witqItem%
            \edef\@witqXItem{\@witqItem}%
            \ifx\@witqXItem\@witqFirst\global\edef\@witqCorrectResponse{{\Alph{enumi}}: \@witqXItem}\fi%
            \stepcounter{enumi}%
            }
        \end{enumerate}
    \end{column}
    \end{columns}
    \end{frame}
}
    

% Internal implementation; this allows one more level of expansion
% for the number and name.
\def\@witqG#1#2#3{
    \typeout{Q #3 #2}%
    \g@addto@macro\answers{\item<#3,\arabic{@qnum}> #2\pause}%
    \g@addto@macro\answerimages{\only<#3>{\noindent\includegraphics[width=\textwidth]{img/\@witqprefix-#1}}}
}

\def\@txtqG#1#2#3{
    \typeout{Q #3 #2}%
    \g@addto@macro\answers{\item<#3,\arabic{@qnum}> #2\pause}%
    \g@addto@macro\answerimages{\only<#3>{#1}}
}

\def\witqend{
\stepcounter{@qnum}

\wititle{\@witqsection}{Answers}

\begin{frame}
\frametitle{\@witqsection\relax\ -- Answers}

\begin{columns}[T]
\begin{column}{0.3\textwidth}
    \begin{enumerate}
    \answers
    \end{enumerate}
\end{column}
\begin{column}{0.7\textwidth}
    \answerimages
\end{column}
\end{columns}

\end{frame}
}

\makeatother
