% Macro's for picture-based quizzes
%
% Use the following macros for a picture-quiz section
%
% \witqstart{abbrev}{name-of-section}
%   Resets the quiz internals and starts numbering from 1.
%   @param name-of-section is used in slide titles
%   @param abbrev is used as prefix for image file names
%
% \witq{answer}{ans.1}{ans.2}{ans.3}{ans.4}{ans.5}
%   Produces a question slide using *abbrev*-*answer*-q as image
%   and the five answers as text. One of the five must match *answer*.
%
% \textq{question-phrase}{answer-letter}{ans.1}..{ans.5}
%   Produces a question slide with *question-text* as the
%   actual question, and 5 answers. Provide the letter of the
%   answer as the second parameter/
%
% \witqend
%   Produces answers slides, now with *abbrev*-*answer* as
%   (full-screen) image for the answer.
%
% Images may be any format supported by \includegraphics

\makeatletter


\newcommand*\answers{}
\newcommand*\answerimages{}
\newcounter{@qnum}

\def\witqstart#1#2{
    \def\@witqprefix{#1}
    \def\@witqsection{#2}
    \renewcommand\answers{}
    \renewcommand\answerimages{}
    \setcounter{@qnum}{0}
}

\def\txtq#1#2#3#4#5#6#7{
    \expandafter\@witqF{#1}{#3}{#4}{#5}{#6}{#7}
    \if#2A\def\@a{#3}%
    \else%
    \if#2B\def\@a{#4}%
    \else%
    \if#2C\def\@a{#5}%
    \else%
    \if#2D\def\@a{#6}%
    \else%
    \if#2E\def\@a{#7}%
    \else%
    \def\@a{?}%
    \fi\fi\fi\fi\fi
    \stepcounter{@qnum}
    \edef\@n{\arabic{@qnum}}%
    \edef\@arg{{#1}{#2 \@a}{\@n}}
    \expandafter\@txtqG\@arg
}

\def\witq#1#2#3#4#5#6{
    \expandafter\@witqF{
        \vspace{-1cm}
        \begin{center}
        \noindent\includegraphics[width=\textwidth]{img/\@witqprefix-#1-q}
        \end{center}
}{#2}{#3}{#4}{#5}{#6}

    \def\@ansX{#1}%
    \def\@ansA{#2}%
    \def\@ansB{#3}%
    \def\@ansC{#4}%
    \def\@ansD{#5}%
    \def\@ansE{#6}%
    \stepcounter{@qnum}
    \edef\@n{\arabic{@qnum}}%
    \ifx\@ansX\@ansA\def\@a{A}%
    \else%
    \ifx\@ansX\@ansB\def\@a{B}%
    \else%
    \ifx\@ansX\@ansC\def\@a{C}%
    \else%
    \ifx\@ansX\@ansD\def\@a{D}%
    \else%
    \ifx\@ansX\@ansE\def\@a{E}%
    \else%
    \def\@a{?}%
    \fi\fi\fi\fi\fi
    \edef\@arg{{#1}{\@a}{\@n}}
    \expandafter\@witqG\@arg
}

% Internal implementation of frame content
\def\@witqF#1#2#3#4#5#6{
    \begin{frame}
    \frametitle{\@witqsection}
    \begin{columns}
    \begin{column}{0.7\textwidth}
        #1
    \end{column}
    \begin{column}{0.3\textwidth}
        \bigskip
        \begin{enumerate}
            \item[A] #2
            \item[B] #3
            \item[C] #4
            \item[D] #5
            \item[E] #6
        \end{enumerate}
    \end{column}
    \end{columns}
    \end{frame}
}

% Internal implementation; this allows one more level of expansion
% for the number and name.
\def\@witqG#1#2#3{
    \g@addto@macro\answers{\item<#3,\arabic{@qnum}> #2 #1\pause}%
    \g@addto@macro\answerimages{\only<#3>{\noindent\includegraphics[width=\textwidth]{img/\@witqprefix-#1}}}
}

\def\@txtqG#1#2#3{
    \g@addto@macro\answers{\item<#3,\arabic{@qnum}> #2\pause}%
    \g@addto@macro\answerimages{\only<#3>{#1}}
}

\def\witqend{
\stepcounter{@qnum}

\begin{frame}
\frametitle{\@witqsection -- Answers}

\begin{columns}[T]
\begin{column}{0.3\textwidth}
    \begin{enumerate}
    \answers
    \end{enumerate}
\end{column}
\begin{column}{0.7\textwidth}
    \answerimages
\end{column}
\end{columns}

\end{frame}
}

\makeatother
